Modifica el modo del canal\hypertarget{unknown_case_SYNOPSIS}{}\section{S\-Y\-N\-O\-P\-S\-I\-S}\label{unknown_case_SYNOPSIS}
{\bfseries \#include} {\bfseries $<$redes2/chat.\-h$>$} 

{\bfseries void} {\bfseries mode\-\_\-case} {\bfseries }({\bfseries char{\bfseries $\ast${\bfseries prefix{\bfseries },} {\bfseries char{\bfseries $\ast${\bfseries channel{\bfseries },{\bfseries char{\bfseries $\ast${\bfseries mode{\bfseries },} {\bfseries char{\bfseries $\ast${\bfseries user{\bfseries },} {\bfseries char{\bfseries $\ast$$\ast${\bfseries nick{\bfseries },} const{\bfseries int{\bfseries sd{\bfseries })}  } } descripcion} D\-E\-S\-C\-R\-I\-P\-C\-IÓ\-N}  Modifica} el} modo} del} canal} si} este} se} especifica, en caso contrario indica el modo del canal.

Recibe como parámetros el prefijo que nos da la funcion de parseo del comando,el canal sobre el cual se ejecuta el modo, el modo del canal, el usuario que solicita el cambio, el nick del usuario que envia el mensaje y su socket correspondiente.\hypertarget{unknown_case_retorno}{}\section{R\-E\-T\-O\-R\-N\-O}\label{unknown_case_retorno}
No devuelve nada.\hypertarget{unknown_case_seealso}{}\section{V\-E\-R T\-A\-M\-B\-IÉ\-N}\label{unknown_case_seealso}
{\bfseries } \hypertarget{unknown_case_authors}{}\section{A\-U\-T\-O\-R}\label{unknown_case_authors}
Francisco Andreu Sanz (\href{mailto:francisco.andreu@estudiante.uam.es}{\tt francisco.\-andreu@estudiante.\-uam.\-es}) Javier Martínez Hernández (\href{mailto:javier.maritnez@estudiante.uam.es}{\tt javier.\-maritnez@estudiante.\-uam.\-es}) 