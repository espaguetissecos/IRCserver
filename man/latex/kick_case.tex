Expulsa al usuario que le indiquemos del canal especificado. Opcionalmente añade un comentario.\hypertarget{unknown_case_SYNOPSIS}{}\section{S\-Y\-N\-O\-P\-S\-I\-S}\label{unknown_case_SYNOPSIS}
{\bfseries \#include} {\bfseries $<$redes2/chat.\-h$>$} 

{\bfseries void} {\bfseries kick\-\_\-case} {\bfseries }({\bfseries char{\bfseries $\ast${\bfseries prefix{\bfseries },} {\bfseries char{\bfseries $\ast${\bfseries channel{\bfseries },} {\bfseries char{\bfseries $\ast${\bfseries user{\bfseries },} {\bfseries char{\bfseries $\ast${\bfseries comment{\bfseries },} {\bfseries char{\bfseries $\ast$$\ast${\bfseries nick{\bfseries },} const{\bfseries int{\bfseries sd{\bfseries })}  } } descripcion} D\-E\-S\-C\-R\-I\-P\-C\-IÓ\-N}  Expulsa} al} usuario} que} le} indiquemos} del} canal especificado. Opcionalmente añade un comentario.

Recibe como parámetros el prefijo que nos da la funcion de parseo del comando,el canal para expulsar al usuario, el usuario que deseamos expulsar, el mensaje que explica el motivo de la expulsión, el nick del usuario que envia el mensaje y su socket correspondiente.\hypertarget{unknown_case_retorno}{}\section{R\-E\-T\-O\-R\-N\-O}\label{unknown_case_retorno}
No devuelve nada.\hypertarget{unknown_case_seealso}{}\section{V\-E\-R T\-A\-M\-B\-IÉ\-N}\label{unknown_case_seealso}
{\bfseries } \hypertarget{unknown_case_authors}{}\section{A\-U\-T\-O\-R}\label{unknown_case_authors}
Francisco Andreu Sanz (\href{mailto:francisco.andreu@estudiante.uam.es}{\tt francisco.\-andreu@estudiante.\-uam.\-es}) Javier Martínez Hernández (\href{mailto:javier.maritnez@estudiante.uam.es}{\tt javier.\-maritnez@estudiante.\-uam.\-es}) 